\documentclass{beamer}
\usepackage{tabularx}
\usepackage{amsmath}
\usepackage{amssymb}
\usepackage{soul}
\mode<presentation>
{
	\usecolortheme{fly}
	\setbeamercovered{transparent}
}
\setbeamertemplate{caption}{\raggedright\insertcaption\par}
\title{BoF SBI extension stuff}
\author{}
\subtitle{}
\date{LPC 2022}

\begin{document}
\begin{frame}
	\titlepage
\end{frame}

\begin{frame}{}
	Andes/Renesas:
	\begin{itemize}
		\item[--]
		They don't have DMA coherent peripherals
		\item[--]
		They used some SBI extensions to configure non-coherency
		\item[--]
		Sent RFC patchset that almost certainly needs to become alternatives if acceptable
		\item[--]
		Touches a bunch of arch code as is, some could (and should) be moved to drivers/soc
		\item[--]
		https://lore.kernel.org/linux-riscv/20220906102154.32526-1-prabhakar.mahadev-lad.rj@bp.renesas.com/
	\end{itemize}
\end{frame}
\begin{frame}{}
	Microchip:
	\begin{itemize}
		\item[--]
		Use an extention for our AMP inter-hart communication as an isolation mechanism
		\item[--]
		Allows e51 to handle all writes to the IP
		\item[--]
		"No" arch code impact - just in a mailbox and a remoteproc driver (like qcom)
		\item[--]
		https://github.com/linux4microchip/linux/blob/linux-5.15-mchp/drivers/mailbox/mailbox-miv-ihc.c
	\end{itemize}
\end{frame}

\begin{frame}{}
Arch policy doc:
https://www.kernel.org/doc/html/latest/riscv/patch-acceptance.html
"To avoid the maintenance complexity and potential performance impact of adding kernel code for implementor-specific RISC-V extensions, we’ll only to accept patches for extensions that have been officially frozen or ratified by the RISC-V Foundation."
\end{frame}
\end{document}
